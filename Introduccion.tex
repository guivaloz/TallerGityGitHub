%
% Taller de Git y GitHub
%
% Introducción
%

\section{Introducción}

\subsection{A cerca de Git}
\begin{frame}
  \frametitle{A cerca de Git}
  \begin{itemize}
    \item<1-> Git es un software de control de revisiones: Es capaz de recordar los estados previos en que se hayan guardado los muchos archivos de un sistema/documentación/sitio.
    \item<2-> Se puede comunicar con computadoras y servidores remotos, para lograr así la descarga, sincronización y actualización.
    \item<3-> Usado por los desarrolladores del kernel \textbf{Linux} y diseñado por \textbf{Linus Torvalds}.
    \item<4-> Es Software Libre con licencia GPL versión 2.
  \end{itemize}
\end{frame}

\subsection{A cerca de GitHub}
\begin{frame}
  \frametitle{A cerca de GitHub}
  \begin{itemize}
    \item<1-> GitHub es el repositorio más grande y popular de software que usa Git.
    \item<2-> Ofrece \textbf{alojamiento ilimitado} en la nube gratuito \textbf{para todo lo que sea abierto}.
    \item<3-> Las \textbf{condiciones de uso} especifican que debe ser un humano de más de 13 años.
    \item<4-> Que uno es \textbf{responsable} de la cuenta y de todo el contenido que subas.
    \item<5-> Que \textbf{NO} es para contenido ilegales o no autorizados.
  \end{itemize}
\end{frame}
